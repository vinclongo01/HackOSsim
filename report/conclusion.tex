\newpage
\section{Conclusions}
Our project aims to show the use of \textbf{FreeRTOS} on \textbf{Qemu} through tutorials, exercises and modifying some solutions of the operating system. 
Some demo applications were developed to provide a comprehensive overview on FreeRTOS' \textbf{task management}, \textbf{queue and semaphore's usage} for \textbf{task-to-task communication and tasks' synchronization}. Next, new implementations of heap allocation algorithms were implemented by revising the \texttt{heap\_4.c} file provided with the FreeRTOS distribution. Particularly, the \textbf{Best-Fit} and \textbf{Worst-Fit} algorithms are added to the \textbf{First-Fit} one, which is the default choice. Finally, a demo application was developed to evaluate the newly implemented solution. Overall, developing this project helps us to gain the main skills in \textbf{FreeRTOS programming}, to better understand \textbf{embedded systems} and \textbf{real time operating systems}. 

\section{Statement Of Work}

The work of each member of the group is summarised below.

\begin{table}[h!]
    \centering
    \begin{tabular}{|c|l|}
        \hline
        \textbf{Member} & \textbf{Work Done} \\ \hline
        Rachid Youssef Grib & 
        \begin{tabular}[c]{@{}l@{}}
            Memory Management, Queue and Tasks Synchronization
        \end{tabular} \\ \hline
        Giuseppe Famà & 
        \begin{tabular}[c]{@{}l@{}}
           Installation, Queue and Tasks Synchronization
        \end{tabular} \\ \hline
        Vincenzo Longo & 
        \begin{tabular}[c]{@{}l@{}}
            Installation, Tasks Management
        \end{tabular} \\ \hline
    \end{tabular}
\end{table}

\noindent The job was done in \textbf{presence}, \textbf{all the participants actively participated} to this project.
